\documentclass[twoside,letterpaper,twocolumn]{article}

%%%%%%%%%%%%%%%%%%%%%%%%%%%%%%%%%%%%%%%%%%%%%%%%%%%%%%%%%%%%%%%%%%%%%%
% Package with format specifications for the 16th CISBGf
\usepackage{cisbgf19}

%%%%%%%%%%%%%%%%%%%%%%%%%%%%%%%%%%%%%%%%%%%%%%%%%%%%%%%%%%%%%%%%%%%%%%
% MANDATORY PARAMETERS

% Setting the title
\title{Latex Template for International Congress of the Brazilian Geophysical Society}

% Setting the authors
\author{Felipe Timóteo da Costa, Adelson Oliveira, Jurandyr Schmidt, Adelson Oliveira, Rodrigo Portugal,Hedison Kiuity Sato}

% Setting the headings
\headauthor{Costa, Felipe Timóteo}
\headtitle{Latex Template}

%%%%%%%%%%%%%%%%%%%%%%%%%%%%%%%%%%%%%%%%%%%%%%%%%%%%%%%%%%%%%%%%%%%%%%
%%%%%%%%%%%%%%%%%%%%%%%%%%%%%%%%%%%%%%%%%%%%%%%%%%%%%%%%%%%%%%%%%%%%%%
\begin{document}

\maketitle

\begin{abstract}

Write the abstract after to finalize the article. The main orientations to writing a good abstract are:

The Abstract is a summary of the study, with the primary emphasis on results and conclusions. Very briefly present the question(s) asked, the experimental design, a summary of observations, and list conclusions. Be very succinct - the abstract should be a single paragraph. It should stand on its own; therefore, do not refer to any other part of the report, such as a figure or table. Avoid long sections of introductory or explanatory material. As a summary of work done, it is written in past tense.

\end{abstract}

\section{Introduction}

After to write the Results section, start to write the Introduction. The main orientations to writing a good introduction are:

\begin{itemize}
	\item Presentation of the problem or research question
	\item Contextualize
	\begin{itemize}
		\item briefly review of the literature to help the reader
		\item to mention the used research methodology 
		\item scope of the study - contextualize the reader about the methods used
	\end{itemize}
	\item The purpose of the study
	\begin{itemize}
		\item explain the scope of the study
		\item explain why some aspects was chosen and other not.
		\item explain why the chosen theory was applied to the data.
		\item avoid a detailed bibliography review
		\item avoid a summary of results 
	\end{itemize}
	\item Establish the objective os the work
\end{itemize}

\begin{figure}[h!]
	\centering
	\subfloat[]{
		\includegraphics[width=0.45\linewidth]{images/logo_congress}
		\label{fig:label1}
	}
	\subfloat[]{
		\includegraphics[width=0.45\linewidth]{images/logo_congress}
		\label{fig:label2}
	}
	\caption{Example using figures side by side a) Figure on the left. b) Figure on the right. }
	\label{fig:label3}
\end{figure}


\subsection{making a citation}

First, you need a bibliography file. Google it to discover how to do it. Tips: The Mendeley program could provide a good solution.

I save the file with the name references.bib in this folder.

Example of indirect citation \citep{Yilmaz2000}.

The book of professor \cite{Claerbout1984} is a good choice for who want to learn more about seismic processing and imaging. This was an example of the direct citation.



\section{Materials and Methods}

This section must be written at first. The main orientations to write a good Materials and Methods are:

\begin{itemize}
	\item Briefly describe the methods used
	\item Cite references to the reader to find more information
	\item Describe the new methods providing sufficient details to others researches can reproduce your experiment
	\item Use subtitles to separate different methodologies
	\item Describe what you did in the past
\end{itemize}

\subsection{Optional subsection}

Bellow is an example of how to write an equation using the Latex's math ambient:

\begin{equation}\label{waveequation}
\frac{ 1}{  c^2(\mathbf{x})}\frac{\partial^2}{\partial t^2}p(\mathbf{x},t) - \nabla^2 p(\mathbf{x},t)  = w(\mathbf{x}_s,t),				
\end{equation}

\noindent
or it possible to use the math symbols inside the text like the following example. The previous wave equation is the 2D wave equation, where $ c(\mathbf{x}) $ is the P wave velocity of the medium. 

Example of an equation using 2 lines:

\begin{eqnarray}\label{unconstrainedLagrangian}
 \begin{array}{rl}
\mathcal{L}(w,p,p^{\dagger}) = &\displaystyle \frac{1}{2}\int_{T} \| \mathbf{d}_{obs}(z_{rec},t) - \mathbf{d}_{cal}(z_{rec},t) \|^2 dt \\
                               &\displaystyle - \int_{T}p^{\dagger}\left[ F\left(p(z,t)\right) - w(t) \right] dt ,
\end{array}   
\end{eqnarray}



\begin{figure}[h!]
	\centering
	\includegraphics[width=0.9\linewidth]{images/logo_congress}
	\label{fig:label4}	
	\caption{Example using a simple figure}
\end{figure}


\section{Results}

The second section that should be written is the result section. The main orientations to write a good results section:

\begin{itemize}
	\item The results must be presented in a logical sequence
	\item Don't duplicate data among figures, tables, and text
	\item Describe all parameters used in the experiments
	\item Provide sufficient information to the other researchers can reproduce the experiment
	\item Use subtitles to separate the different experiments
	
\end{itemize}


\begin{figure}[h!]
	\centering
	\subfloat[]{
		\includegraphics[width=0.45\linewidth]{images/logo_congress}
		\label{fig:label5}
	}
	\subfloat[]{
		\includegraphics[width=0.45\linewidth]{images/logo_congress}
		\label{fig:label6}
	}\\
	\subfloat[]{
		\includegraphics[width=0.45\linewidth]{images/logo_congress}
		\label{fig:label7}
	}
	\subfloat[]{
		\includegraphics[width=0.45\linewidth]{images/logo_congress}
		\label{fig:label8}
	}
	\caption{ Example using 4 figures at once. a) Above on the left b)Above on the right  c)Bellow on the left d) Bellow on the right.}
	\label{fig:label9}
\end{figure}


An example of table

\begin{table}[htb]
	\caption{Descrição tabela.}
	\label{tab:label1}
	\begin{tabular}{|c|c|c|c|}
		\hline 
		\textbf{Methods} & \textbf{column 1} & \textbf{column 2} & \textbf{column 3} \\ 
		\hline 
		Seismological    &  &  &  \\ 
		\hline 
		Gravitational    &  &  &  \\ 
		\hline 
		Electric         &  &  &  \\ 
		\hline 
		Magnetic         &  &  &  \\ 
		\hline 
		Electromagnetic  &  &  &  \\ 
		\hline 
		Thermal          &  &  &  \\ 
		\hline 
	\end{tabular} 
\end{table}


\section*{Discussion and Conclusion}

After to write the introduction, start to write the discussion and conclusion section. You could separate it in two or not. The main orientation are:

\begin{itemize}
	\item Briefly describe the limitation of the study showing the you considered the weakness of your experiments
	\item Discuss the conclusion from most important to less important
	\item Highlight the most important findings in a few words
	\item Relate the results with the hypothesis
	\item  Identify methodological proceedings with relevant results
	\item If your finding are preliminaries, suggest new experiments
	
\end{itemize}

\bibliography{references}

%%% Uncomment if you don't have references bib file
%\section{References}

%Barsh, R. (2009) Web-based technology for storage and pro- cessing of multi-component data in seismology. First steps towards a new design, PhD thesis, Ludwig Maximilians Universitat Munchen, Munich, Germany, 126 pp.
%
%Percivall, G. (2010). The application of open standards to en- hance the interoperability of geoscience information, In- ternational Journal of Digital Earth, 3 (S1), 14-30.
%
%Pirchiner, M ; Collaço, B. B. ; Calhau, J. ; Assumpção, M. S. ; Dourado, J. C. . BRAzilian Seismographic Integrated Systems (BRASIS): infrastructure and data management. Annals of Geophysics, v. 54, p. 17-22, 2011.
%
%Schorlemmer, D., A. Wyss, S. Maraini, S. Wiemer and M. Baer (2004). QuakeML - an XML schema for seismology, ORFEUS Newsletter, 6 (2); http://www.orfeus-eu.org/Organiza- tion/Newsletter/vol6no2/quakeml.shtml (last accessed 7 March 2011).
%
%SEED Manual (2010). SEED Reference Manual, SEED Format Version 2.4, May 2010, IRIS, 212 pp.; 
%
%SeisComP3.org (2011). SeisComP 3 documentation; www.seiscomp3.org/wiki/doc (last accessed 20 April 2013).

\section{Acknowledgments}

\lipsum[11]

\end{document}


